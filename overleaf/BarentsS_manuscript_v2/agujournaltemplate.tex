%%%%%%%%%%%%%%%%%%%%%%%%%%%%%%%%%%%%%%%%%%%%%%%%%%%%%%%%%%%%%%%%%%%%%%%%%%%%
% AGUJournalTemplate.tex: this template file is for articles formatted with LaTeX
%
% This file includes commands and instructions
% given in the order necessary to produce a final output that will
% satisfy AGU requirements, including customized APA reference formatting.
%
% You may copy this file and give it your
% article name, and enter your text.
%
% guidelines and troubleshooting are here: 

%% To submit your paper:
\documentclass[draft]{agujournal2019}
\usepackage{url} %this package should fix any errors with URLs in refs.
\usepackage{lineno}
\usepackage[inline]{trackchanges} %for better track changes. finalnew option will compile document with changes incorporated.
\usepackage{soul}
\usepackage{wrapfig}
\linenumbers
%%%%%%%
% As of 2018 we recommend use of the TrackChanges package to mark revisions.
% The trackchanges package adds five new LaTeX commands:
%
%  \note[editor]{The note}
%  \annote[editor]{Text to annotate}{The note}
%  \add[editor]{Text to add}
%  \remove[editor]{Text to remove}
%  \change[editor]{Text to remove}{Text to add}
%
% complete documentation is here: http://trackchanges.sourceforge.net/
%%%%%%%

\draftfalse

%% Enter journal name below.
%% Choose from this list of Journals:
%
% JGR: Atmospheres
% JGR: Biogeosciences
% JGR: Earth Surface
% JGR: Oceans
% JGR: Planets
% JGR: Solid Earth
% JGR: Space Physics
% Global Biogeochemical Cycles
% Geophysical Research Letters
% Paleoceanography and Paleoclimatology
% Radio Science
% Reviews of Geophysics
% Tectonics
% Space Weather
% Water Resources Research
% Geochemistry, Geophysics, Geosystems
% Journal of Advances in Modeling Earth Systems (JAMES)
% Earth's Future
% Earth and Space Science
% Geohealth
%
% ie, \journalname{Water Resources Research}

\journalname{Enter journal name here}


\begin{document}

%%%%%%%%%%%%%%%%%%%%%%%%%%%%%%%%%%%%%%%%%%%%%%%
%  TITLE
%
% (A title should be specific, informative, and brief. Use
% abbreviations only if they are defined in the abstract. Titles that
% start with general keywords then specific terms are optimized in
% searches)
%
%%%%%%%%%%%%%%%%%%%%%%%%%%%%%%%%%%%%%%%%%%%%%%%

% Example: \title{This is a test title}

\title{=enter title here=}

%%%%%%%%%%%%%%%%%%%%%%%%%%%%%%%%%%%%%%%%%%%%%%%
%
%  AUTHORS AND AFFILIATIONS
%
%%%%%%%%%%%%%%%%%%%%%%%%%%%%%%%%%%%%%%%%%%%%%%%

% Authors are individuals who have significantly contributed to the
% research and preparation of the article. Group authors are allowed, if
% each author in the group is separately identified in an appendix.)

% List authors by first name or initial followed by last name and
% separated by commas. Use \affil{} to number affiliations, and
% \thanks{} for author notes.
% Additional author notes should be indicated with \thanks{} (for
% example, for current addresses).

% Example: \authors{A. B. Author\affil{1}\thanks{Current address, Antartica}, B. C. Author\affil{2,3}, and D. E.
% Author\affil{3,4}\thanks{Also funded by Monsanto.}}

\authors{=list all authors here=}


% \affiliation{1}{First Affiliation}
% \affiliation{2}{Second Affiliation}
% \affiliation{3}{Third Affiliation}
% \affiliation{4}{Fourth Affiliation}

\affiliation{=number=}{=Affiliation Address=}
%(repeat as many times as is necessary)


% Corresponding author mailing address and e-mail address:

% (include name and email addresses of the corresponding author.  More
% than one corresponding author is allowed in this LaTeX file and for
% publication; but only one corresponding author is allowed in our
% editorial system.)

% Example: \correspondingauthor{First and Last Name}{email@address.edu}

\correspondingauthor{=name=}{=email address=}



%%%%%%%%%%%%%%%%%%%%%%%%%%%%%%%%%%%%%%%%%%%%%%%
% KEY POINTS
%%%%%%%%%%%%%%%%%%%%%%%%%%%%%%%%%%%%%%%%%%%%%%%
%  List up to three key points (at least one is required)
%  Key Points summarize the main points and conclusions of the article
%  Each must be 140 characters or fewer with no special characters or punctuation and must be complete sentences

% Example:
% \begin{keypoints}
% \item	List up to three key points (at least one is required)
% \item	Key Points summarize the main points and conclusions of the article
% \item	Each must be 140 characters or fewer with no special characters or punctuation and must be complete sentences
% \end{keypoints}

\begin{keypoints}
\item enter point 1 here
\item enter point 2 here
\item enter point 3 here
\end{keypoints}

%%%%%%%%%%%%%%%%%%%%%%%%%%%%%%%%%%%%%%%%%%%%%%%
%
%  ABSTRACT and PLAIN LANGUAGE SUMMARY
%
% A good Abstract will begin with a short description of the problem
% being addressed, briefly describe the new data or analyses, then
% briefly states the main conclusion(s) and how they are supported and
% uncertainties.

% The Plain Language Summary should be written for a broad audience,
% including journalists and the science-interested public, that will not have 
% a background in your field.
%
% A Plain Language Summary is required in GRL, JGR: Planets, JGR: Biogeosciences,
% JGR: Oceans, G-Cubed, Reviews of Geophysics, and JAMES.
% see http://sharingscience.agu.org/creating-plain-language-summary/)
%
%%%%%%%%%%%%%%%%%%%%%%%%%%%%%%%%%%%%%%%%%%%%%%%

%% \begin{abstract} starts the second page

\begin{abstract}
[ enter your Abstract here ]
\end{abstract}

\section*{Plain Language Summary}
Enter your Plain Language Summary here or delete this section.
Here are instructions on writing a Plain Language Summary: 
https://www.agu.org/Share-and-Advocate/Share/Community/Plain-language-summary


%%%%%%%%%%%%%%%%%%%%%%%%%%%%%%%%%%%%%%%%%%%%%%%
%
%  BODY TEXT
%
%%%%%%%%%%%%%%%%%%%%%%%%%%%%%%%%%%%%%%%%%%%%%%%

%%% Suggested section heads:
% \section{Introduction}
%
% The main text should start with an introduction. Except for short
% manuscripts (such as comments and replies), the text should be divided
% into sections, each with its own heading.

% Headings should be sentence fragments and do not begin with a
% lowercase letter or number. Examples of good headings are:

% \section{Materials and Methods}
% Here is text on Materials and Methods.
%
% \subsection{A descriptive heading about methods}
% More about Methods.
%
% \section{Data} (Or section title might be a descriptive heading about data)
%
% \section{Results} (Or section title might be a descriptive heading about the
% results)
%
% \section{Conclusions}

\section{Introduction}\label{intro}
% examples of citation formatting: 
% \cite{Arthun2012}
% \citeA{Arthun2012}

% global context, why are we interested in the Arctic, Arctic Amplification; regional feedback mechanisms with sea ice loss, how does Arctic circulation, heat fluxes and water masses require use to understand transformation process
% FIGURE 1 to include: an Arctic map of Bathymetry or two of heat/freshwater content anomalies
% mean velocity of the Barents Sea AW layer to show the direction of flow
% start big, make this smaller and smaller until the Barents Sea
It is well-established that the Arctic is warming at a substantially faster rate than anywhere else on Earth in a process known as Arctic Amplification (AA) \cite{Manabe1980,Serreze2009,Cosimo2014,Huang2017,Rantanen2022}. This amplification results from several interrelated feedback mechanisms, though the relative contributions of each are still debated \cite{Pithan2014,Timmermans2018,Gong2017,Pistone2019,Previdi2021}. One important driver of AA is the rapid decline in winter sea ice extent and thickness \cite{Perovich2009,Dai2019}, closely linked to ocean heat transport and changes in surface heat forcing \cite{Onarheim2018,Stroeve2018,Oldenburg2024}. The implications of this amplification extend beyond the Arctic; warming in the lower troposphere over the Arctic can potentially influence mid-latitude weather patterns \cite{Honda2009,Petoukhov2010,Francis2012,Cohen2018,Coumou2018}. Reduction in sea ice extent is driving changes in migratory patterns in a process called borealization, which impacts local ecosystems and local economies \cite{Fossheim2015,Polyakov2020_borealization,Ingvaldsen2021}. AA is not uniform, nor is the warming of the ocean explained by the changing atmosphere alone \cite{Marshall2014}. The Northern Barents Sea is a particularly interesting focal point for this change, where winter sea ice decline and surface air temperature (SAT) amplification are among the highest in the Arctic \cite{Screen2010,Onarheim2017,Isaksen2022,Rantanen2022}. Warming in this region penetrates the water column \cite{Smedsrud2013} and reduces sea ice extent \cite{Onarheim2018,Previdi2021}; these changes are altering stratification and making the sea more dynamically similar to the North Atlantic, signaling a possible regime shift \cite{Arthun2012,Lind2018,Skagseth2020}.

% structure of the Barents Sea - depth, inflow and outflow, etc
% add more detail about the structure of the sea; weird surface stratification which is unique to this region
% ATLANTIFICATION as a part of amplification; role of Advection and sea ice dynamics on WMT; explain advection, transport of heat and salt through advective transport, how AW drives transformation during this process; how do sea ice dynamics affect vertical mixing and alter mixing and water mass properties
Previous studies have hypothesized at the feedback mechanisms in the Barents Sea, including the relatedness of amplification, ocean heat transport (OHT), and acute sea ice reduction \cite{Screen2010,Arthun2012,Stroeve2018,Previdi2021}. The influence of Atlantic Water (AW) is  increasingly important in driving ice reduction through OHT, a process coined "Atlantification" \cite{Polyakov2017,Arthun2019}. While OHT is largely responsible for the reduction in sea ice, and the sea ice loss intensifies AA, neither of these fully explains observations \cite{Screen2010,Smedsrud2013} or model simulations \cite{Pithan2014,Li2017}. Here our focus is not on these individual processes alone; this complexity underscores the need to disentangle the relative roles of these terms on heat and freshwater modification.

% barents sea structure
The Barents Sea is a comparatively shallow gateway to the Arctic Ocean and is shaped by two primary water masses: AW and Arctic Water, defined by their temperature (\emph{T}) and salinity (\emph{S}). Relatively warm and saline, AW enters through the Barents Sea Opening (BSO), cooling and freshening as it flows northward before exiting towards the Arctic Ocean at depth \cite{Loeng1991,Smedsrud2010}. In the North, westward and southward currents carry colder and fresher Arctic Water towards the polar front. The position of this polar front characterizes the Barents Sea into two distinct regimes: the southern part dominated by Atlantic Water (AW) inflow \cite{Hakkinen2009} and the northern part marked by seasonal sea ice coverage and colder, more Arctic conditions \cite{kolas2024}. Following the alpha/beta ocean concept \cite{Nansen1902,Carmack2007,Stewart2016}, the southern Barents Sea resembles the Atlantic with a temperature-stratified "alpha" structure, while the northern salt-stratified "beta" regime is maintained by sea ice melt. In the south, AW flows into the sea at depth, and as it progresses northward, is prevented from surfacing by a fresh, near-surface Arctic Water layer. With Atlantification, warm water is spreading northwards, threatening local ecologies \cite{Bogstad2015,Dalpadado2014,Ingvaldsen2021} and weakening the stratification of the sea itself \cite{Lind2018}. Given that AW holds enough heat to melt Arctic sea ice were it to surface and has broader implications for the Arctic Ocean warming, understanding the mechanisms by which it interacts with Arctic Water and how this alters the stratification is essential \cite{Polyakov2017,Stroeve2018,Skagseth2020,Grabon2021}.

% 47: However, recent studies have highlighted the relationship between amplified hotspot in the North with increasing heat advected through the Barents Sea Opening (BSO), contributing to sea ice retreat \cite{Arthun2012,Li2017} as well as weaker stratification and enhanced vertical mixing and upward heat transfer \cite{Lind2018,Gerland2023}, limiting sea ice formation \cite{Barton18,Wang2019}. 

% 52-53: This phenomenon is referred to as "Atlantification", has resulted in a northward expansion of the more Atlantic regime of the Barents Sea, with implications for the local ecology \cite{Bogstad2015,Dalpadado2014,Ingvaldsen2021}. Furthermore, because the heat carried by AW is enough to melt the sea ice of the internal Arctic were it to reach the surface, the Atlantification of the Barents Sea has resulted in warmed outflows, with implications for Arctic-wide transformations \cite{Polyakov2017,Skagseth2020,Grabon2021}. 

% 60: As the Barents Sea becomes a more Atlantic-dominated regime, its capacity for cooling is declining, with consequences for the heat carried to the Arctic \cite{Smedsrud2022}. Thus, it is essential to understand how surface forcing, Atlantic inflow, and internal mixing jointly contribute to heat and freshwater variability, shaping both regional and Arctic-wide transformations.

%%%%%%%%%%%%%%%%%%%%%

% physical changes in the Barents Sea: observed trends, recent studies, sea ice extent and seasonal variability, focus on how reduced sea ice interacts with advective inflow; ongoing WMT including conversion of AW to Arctic Water through heat loss with transport through Barents Sea
% FIGURE 2: the two regimes identified in map view, a comparison of theta change with depth
% timeseries of sea ice extent, freshwater content of Barents Sea, theta with depth
(DRAFT) The upper ocean heat content of of the Barents sea increased steadily from 2003--2017 (Fig~\ref{fig:timeseries}). The upper ocean freshwater content decreased by 50\% during the first five years of this period, remained steady until 2013, then increased from 2013--2017. At the same time, sea ice extent has been steadily declining throughout this period, with minima in 2012 and 2016.

\begin{wrapfigure}{r}{0.5\textwidth}
    \includegraphics[width=\linewidth]{figs/Arctic_timeseries_proposal.png}
    \caption{This is an example caption that will be changed later...}
    \label{fig:timeseries}
\end{wrapfigure}


% paragraph on WMT -- what it is and why do we care
% summarizee some insights provided by earlier studies on why the WMT framework is useful
One method to capture the shifting thermohaline stratification in is by the use of the water mass transformation (WMT) framework. Water masses, defined by discrete sets of \emph{T} and  \emph{S} coordinates, have long been used to study the layering and large scale circulation in the ocean because of their relation to ocean density \cite{sverdrup1942}. In this framework, "transformations" refer to the change of one ocean water mass to another, typically through these alterations in \emph{T} and \emph{S} due to processes like advection, internal diffusion, and surface fluxes. Using the equations of continuity, the observed WMT can be related to these discrete processes. In the first paper of its kind, \citeA{Walin_1977} demonstrated how transformations in salinity can be related to flux- and mixing-driven volume transport across isohaline surfaces. In a subsequent paper, \citeA{Walin1982} showed that volume flow across isothermal surfaces can be used to study the impact of advective and diffusive heat fluxes at the ocean surface on the temperature distribution of the ocean, thus enabling water mass formation calculations from air-sea fluxes. \citeA{Tziperman1986} applied these ideas in density, studying WMT across isopycnals due to internal diffusion and surface heat fluxes. \citeA{Speer1993} synthesized the ideas from \citeA{Walin_1977} and \citeA{Walin1982} and applied this to describe transformations in \emph{S} and \emph{T} coordinates, using the continuity equations to show how the thermodynamic budgets for heat and freshwater can be used to create a two-dimensional conversion vector that describes conversion rate of one water mass into another induced by surface heat and freshwater fluxes. Using this, \citeA{Speer1993} also made it possible to calculate the convergence, identifying the total amount of accumulation or destruction of a given water mass due to surface processes, but did not fully separate the contributions of various processes.

% give some sample studies of WMT in the Arctic Ocean
Building on these foundational papers, a thermohaline streamfunction has been introduced by \citeA{doos2012} and \citeA{zika2012}, representing the global ocean circulation in \emph{T}--\emph{S} coordinates under the assumption of volume conservation. \citeA{zika2012} highlighted that, in a steady state, transformation vectors (e.g. \citeA{Speer1993}) show no net convergence. \citeA{Hieronymus2014} further refined the WMT framework by formulating a continuity equation in \emph{T}--\emph{S} space, enabling the decomposition of WMT drivers into process-specific terms; namely, he added to the \citeA{Speer1993} setup the isoneutral and dianeutral mixing. This further allowed for the attribution of WMT to specific processes, but did not include convection or bottom boundary terms, and thus did not close a budget. In the Arctic Ocean, the WMT framework from \citeA{Hieronymus2014} was applied to long-term mean transformations to study the interactions between Atlantic and Pacific inflows as well as ice dynamics by \citeA{Pemberton2015}. While this application provided valuable insights into dominant, long-term transformation trends, it did not close the budget nor address shorter-term processes, leaving gaps in understanding the relative impacts of various transformation mechanisms in Arctic WMT.

% paragraph on budgets -- motivate this
% L98 need to explain why previous studies have residuals (i.e., why are budgets not closed). Similarly it would be helpful to broadly summarize what “various terms” contribute to the WMT budget
% L102 here it should be stated that you investigate WMT in ASTE and why ASTE is an appropriate/advantageous framework to use
% L103 there is only one continuity equation. The budgets are not derived solely from continuity…
% L105 “ascertain the attribution of both…” needs to be rewritten 
% L106-108: this argument is confusing. Changes in T and S can be “connected…to variations in density” without the WMT framework (i.e., just via the equation of state)
% L110 I do not follow the emphasis on enhanced AW role in northern Barents Sea
% L111 not clear from what’s written why heat fluxes and ice retreat are necessarily “reinforcing”
% L114 bit confusing to say “regime shifts” here if you mean temporal trends, since regime differences have been described above in reference to south-north transitions through Barents Sea
% L116 hypothesis should be given for science questions here, not for application of method, which has already been established (i.e., we know this will work)
% L120 headings for “Budget analysis” and “T-S Analysis” not quite right. Are you thinking to describe heat and salt budgets in the former and WMT in T-S space (also a budget) in the latter?

% paragraph on T--S analysis; why is this important: discuss in relation to the alpha and beta oceans
The application of \emph{T}--\emph{S} coordinates together can be useful to study WMT because these terms contribute to ocean density. Barents Sea is a transition zone between the alpha and beta oceans as described by \cite{Stewart2016}. 

Studying temperature alone, while it presents a good idea of the surface heat transformations, does not allow us to study the differecnes between beta/alpha oceans which are salt-stratified.

Transformations in \emph{S} coordinates are important in the Arctic as they capture surface freshwater inputs (e.g. runoff, brine rejection) and given salt has a greater impact on density than temperature these can be used to approximate density changes, but in the beta/alpha (whichever) it is important to use both, the profile of temperature is atypical. 


% % paragraph introducing/motivating WMT in the Barents Sea/how has this been applied in other studies
% As the heat and salinity of the Barents Sea change, so too does its buoyancy. Ocean stratification is determined by buoyancy---a crucial term in the vertical momentum equation. Positive buoyancy drives upward acceleration, while negative buoyancy causes downward motion, influencing the stability of the water column. In a stratified ocean, buoyancy depends on density, which acts to restore displaced fluid particles to maintain equilibrium. In seawater, fluid density ($\rho$) depends on temperature and salt, but salinity changes exert a far greater impact on $\rho$. This makes salinity an important factor shaping stratification, particularly in polar regions where surface salt fluxes are substantial.


% WMT in the Barents Sea -- what's the point
We employ a fully-closed WMT framework in \emph{T}--\emph{S} space for the Barents Sea. Closed budgets, derived from continuity equations, are essential for accurately capturing heat and freshwater variability by accounting for all contributing terms. The elimination of residuals thus allows us to ascertain the attribution of both dominant and minor physical processes on \emph{T} and \emph{S} changes. In the context of Atlantification in the Barents Sea, this framework is particularly valuable as it connects the changes in \emph{T} and \emph{S} to variations in density. The advection of AW is expected to play an important role in the observed warming and salinification of the Barents Sea, particularly in the northern, salinity-stratified regime. Surface heat fluxes and sea ice retreat further modulate surface transformation, reinforcing changes in the water column. Internal mixing contributes to WMT, homogenizing the previous contributions from advection and surface processes. Ultimately, the closed-budget WMT framework provides a comprehensive and nuanced understanding of the processes driving regime shifts in the Barents Sea, shedding light on the interplay between these processes.


% motivation and research questions; key gaps in knowledge; goals of the paper as quantifying WMT, identifying key drivers, and understanding their implications on the Arctic system
(DRAFT) We hypothesize that the relative roles of AW import and surface process changes can be quantified using a budgeted WMT framework.


\section{Methodology}\label{methods}

\subsection{Model Description}

\subsection{Budget Analysis}

% why is our budget analysis unique

% what is budget analysis

% INCLUDE the equations for this

% FIGURE 3: demonstrate budget analysis in T--S space

\subsection{Temperature--Salinity Analysis}

% FIGURE 4: for the 5 years; motivate this with figure 1 (SI extent timeseries)
% goal of this is to show that the advective term is changing over time as a contribution to heating and salting

\section{Results}

% FIGURE 5: highlight the convergence anomalies for the five years; compare this to figure 2 with the vertical stratification (redistribution away from the center to make a more average looking profile)
% focus on the one month (march) or maybe take a winter average motivated by the literature, use this for 2012 and 2014 to show that less sea ice -- less stratification -- redistribution of water away from the mixing line


%%

%  Numbered lines in equations:
%  To add line numbers to lines in equations,
%  \begin{linenomath*}
%  \begin{equation}
%  \end{equation}
%  \end{linenomath*}



%% Enter Figures and Tables near as possible to where they are first mentioned:
%
% DO NOT USE \psfrag or \subfigure commands.
%
% Figure captions go below the figure.
% Acronyms used in figure captions will be spelled out in the final, published version.

% Table titles go above tables;  other caption information
%  should be placed in last line of the table, using
% \multicolumn2l{$^a$ This is a table note.}
% NOTE that there is no difference between table caption and table heading in the final, published version
%
%----------------
% EXAMPLE FIGURES
%
% \begin{figure}
% \includegraphics{example.png}
% \caption{caption}
% \end{figure}
%
% Giving latex a width will help it to scale the figure properly. A simple trick is to use \textwidth. Try this if large figures run off the side of the page.
% \begin{figure}
% \noindent\includegraphics[width=\textwidth]{anothersample.png}
%\caption{caption}
%\label{pngfiguresample}
%\end{figure}
%
%
% If you get an error about an unknown bounding box, try specifying the width and height of the figure with the natwidth and natheight options. This is common when trying to add a PDF figure without pdflatex.
% \begin{figure}
% \noindent\includegraphics[natwidth=800px,natheight=600px]{samplefigure.pdf}
%\caption{caption}
%\label{pdffiguresample}
%\end{figure}
%
%
% PDFLatex does not seem to be able to process EPS figures. You may want to try the epstopdf package.
%

%
% ---------------
% EXAMPLE TABLE
%
% \begin{table}
% \caption{Time of the Transition Between Phase 1 and Phase 2$^{a}$}
% \centering
% \begin{tabular}{l c}
% \hline
%  Run  & Time (min)  \\
% \hline
%   $l1$  & 260   \\
%   $l2$  & 300   \\
%   $l3$  & 340   \\
%   $h1$  & 270   \\
%   $h2$  & 250   \\
%   $h3$  & 380   \\
%   $r1$  & 370   \\
%   $r2$  & 390   \\
% \hline
% \multicolumn{2}{l}{$^{a}$Footnote text here.}
% \end{tabular}
% \end{table}

%%%%%%%%%%%%%%%%%%%%%%%%%%%%%%%%%%%%%%%%%%%%%%%
% SIDEWAYS FIGURES and TABLES
% AGU prefers the use of {sidewaystable} over {landscapetable} as it causes fewer problems.
%
% \begin{sidewaysfigure}
% \includegraphics[width=20pc]{figsamp}
% \caption{caption here}
% \label{newfig}
% \end{sidewaysfigure}
%
%  \begin{sidewaystable}
%  \caption{Caption here}
% \label{tab:signif_gap_clos}
%  \begin{tabular}{ccc}
% one&two&three\\
% four&five&six
%  \end{tabular}
%  \end{sidewaystable}

%% If using numbered lines, please surround equations with \begin{linenomath*}...\end{linenomath*}
%\begin{linenomath*}
%\begin{equation}
%y|{f} \sim g(m, \sigma),
%\end{equation}
%\end{linenomath*}

%%% End of body of article

%%%%%%%%%%%%%%%%%%%%%%%%%%%%%%%%%%%%%%%%%%%%%%%
%% Optional Appendices go here
%
% The \appendix command resets counters and redefines section heads
%
% After typing \appendix
%
%\section{Here Is Appendix Title}
% will show
% A: Here Is Appendix Title
%
%\appendix
%\section{Here is a sample appendix}

%%%%%%%%%%%%%%%%%%%%%%%%%%%%%%%%%%%%%%%%%%%%%%%
% Optional Glossary, Notation or Acronym section goes here:
%
% Glossary is only allowed in Reviews of Geophysics
%  \begin{glossary}
%  \term{Term}
%   Term Definition here
%  \term{Term}
%   Term Definition here
%  \term{Term}
%   Term Definition here
%  \end{glossary}


%%%%%%%%%%%%%%%%%%%%%%%%%%%%%%%%%%%%%%%%%%%%%%%
% Acronyms
%% NOTE that acronyms in the final published version will be spelled out when used in figure captions.
%   \begin{acronyms}
%   \acro{Acronym}
%   Definition here
%   \acro{EMOS}
%   Ensemble model output statistics
%   \acro{ECMWF}
%   Centre for Medium-Range Weather Forecasts
%   \end{acronyms}


%%%%%%%%%%%%%%%%%%%%%%%%%%%%%%%%%%%%%%%%%%%%%%%
% Notation
%   \begin{notation}
%   \notation{$a+b$} Notation Definition here
%   \notation{$e=mc^2$}
%   Equation in German-born physicist Albert Einstein's theory of special
%  relativity that showed that the increased relativistic mass ($m$) of a
%  body comes from the energy of motion of the body—that is, its kinetic
%  energy ($E$)—divided by the speed of light squared ($c^2$).
%   \end{notation}




%%%%%%%%%%%%%%%%%%%%%%%%%%%%%%%%%%%%%%%%%%%%%%%
%
% DATA SECTION and ACKNOWLEDGMENTS
%
%%%%%%%%%%%%%%%%%%%%%%%%%%%%%%%%%%%%%%%%%%%%%%%

\section*{Open Research Section}
The ASTE\_R1 model configuration, inputs, and monthly and daily outputs are available at the Arctic Data Center (https://arcticdata.io).


\acknowledgments
Enter acknowledgments here. This section is to acknowledge funding, thank colleagues, enter any secondary affiliations, and so on.


%%%%%%%%%%%%%%%%%%%%%%%%%%%%%%%%%%%%%%%%%%%%%%%
% REFERENCES and BIBLIOGRAPHY
%
\bibliography{agusample} %don't specify the file extension
% don't specify bibliographystyle
%
%%%%%%%%%%%%%%%%%%%%%%%%%%%%%%%%%%%%%%%%%%%%%%%

%\bibliography{ enter your bibtex bibliography filename here }



%Reference citation instructions and examples:
%
% Please use ONLY \cite and \citeA for reference citations.
% \cite for parenthetical references
% ...as shown in recent studies (Simpson et al., 2019)
% \citeA for in-text citations
% ...Simpson et al. (2019) have shown...
%
%
%...as shown by \citeA{jskilby}.
%...as shown by \citeA{lewin76}, \citeA{carson86}, \citeA{bartoldy02}, and \citeA{rinaldi03}.
%...has been shown \cite{jskilbye}.
%...has been shown \cite{lewin76,carson86,bartoldy02,rinaldi03}.
%... \cite <i.e.>[]{lewin76,carson86,bartoldy02,rinaldi03}.
%...has been shown by \cite <e.g.,>[and others]{lewin76}.
%
% apacite uses < > for prenotes and [ ] for postnotes
% DO NOT use other cite commands (e.g., \citet, \citep, \citeyear, \nocite, \citealp, etc.).
%



\end{document}



More Information and Advice:

%%%%%%%%%%%%%%%%%%%%%%%%%%%%%%%%%%%%%%%%%%%%%%%
%
%  SECTION HEADS
%
%%%%%%%%%%%%%%%%%%%%%%%%%%%%%%%%%%%%%%%%%%%%%%%

% Capitalize the first letter of each word (except for
% prepositions, conjunctions, and articles that are
% three or fewer letters).

% AGU follows standard outline style; therefore, there cannot be a section 1 without
% a section 2, or a section 2.3.1 without a section 2.3.2.
% Please make sure your section numbers are balanced.
% ---------------
% Level 1 head
%
% Use the \section{} command to identify level 1 heads;
% type the appropriate head wording between the curly
% brackets, as shown below.
%
%An example:
%\section{Level 1 Head: Introduction}
%
% ---------------
% Level 2 head
%
% Use the \subsection{} command to identify level 2 heads.
%An example:
%\subsection{Level 2 Head}
%
% ---------------
% Level 3 head
%
% Use the \subsubsection{} command to identify level 3 heads
%An example:
%\subsubsection{Level 3 Head}
%
%---------------
% Level 4 head
%
% Use the \subsubsubsection{} command to identify level 3 heads
% An example:
%\subsubsubsection{Level 4 Head} An example.
%
%%%%%%%%%%%%%%%%%%%%%%%%%%%%%%%%%%%%%%%%%%%%%%%
%
%  IN-TEXT LISTS
%
%%%%%%%%%%%%%%%%%%%%%%%%%%%%%%%%%%%%%%%%%%%%%%%
%
% Do not use bulleted lists; enumerated lists are okay.
% \begin{enumerate}
% \item
% \item
% \item
% \end{enumerate}
%
%%%%%%%%%%%%%%%%%%%%%%%%%%%%%%%%%%%%%%%%%%%%%%%
%
%  EQUATIONS
%
%%%%%%%%%%%%%%%%%%%%%%%%%%%%%%%%%%%%%%%%%%%%%%%

% Single-line equations are centered.
% Equation arrays will appear left-aligned.

Math coded inside display math mode \[ ...\]
 will not be numbered, e.g.,:
 \[ x^2=y^2 + z^2\]

 Math coded inside \begin{equation} and \end{equation} will
 be automatically numbered, e.g.,:
 \begin{equation}
 x^2=y^2 + z^2
 \end{equation}


% To create multiline equations, use the
% \begin{eqnarray} and \end{eqnarray} environment
% as demonstrated below.
\begin{eqnarray}
  x_{1} & = & (x - x_{0}) \cos \Theta \nonumber \\
        && + (y - y_{0}) \sin \Theta  \nonumber \\
  y_{1} & = & -(x - x_{0}) \sin \Theta \nonumber \\
        && + (y - y_{0}) \cos \Theta.
\end{eqnarray}

%If you don't want an equation number, use the star form:
%\begin{eqnarray*}...\end{eqnarray*}

% Break each line at a sign of operation
% (+, -, etc.) if possible, with the sign of operation
% on the new line.

% Indent second and subsequent lines to align with
% the first character following the equal sign on the
% first line.

% Use an \hspace{} command to insert horizontal space
% into your equation if necessary. Place an appropriate
% unit of measure between the curly braces, e.g.
% \hspace{1in}; you may have to experiment to achieve
% the correct amount of space.


%%%%%%%%%%%%%%%%%%%%%%%%%%%%%%%%%%%%%%%%%%%%%%%
%
%  EQUATION NUMBERING: COUNTER
%
%%%%%%%%%%%%%%%%%%%%%%%%%%%%%%%%%%%%%%%%%%%%%%%

% You may change equation numbering by resetting
% the equation counter or by explicitly numbering
% an equation.

% To explicitly number an equation, type \eqnum{}
% (with the desired number between the brackets)
% after the \begin{equation} or \begin{eqnarray}
% command.  The \eqnum{} command will affect only
% the equation it appears with; LaTeX will number
% any equations appearing later in the manuscript
% according to the equation counter.
%

% If you have a multiline equation that needs only
% one equation number, use a \nonumber command in
% front of the double backslashes (\\) as shown in
% the multiline equation above.

% If you are using line numbers, remember to surround
% equations with \begin{linenomath*}...\end{linenomath*}

%  To add line numbers to lines in equations:
%  \begin{linenomath*}
%  \begin{equation}
%  \end{equation}
%  \end{linenomath*}



